\documentclass{beamer}

\usepackage[utf8x]{inputenc}
\usepackage{amssymb}
\usepackage{amsmath}
\usepackage{xcolor}
\usepackage{graphicx}
\usepackage{booktabs}
\usepackage{hyperref}
\usepackage[francais]{babel}

\graphicspath{{./images/}}

\usetheme{metropolis}
\title{Machine learning \& expérimentations créatives}
\date{17 septembre 2018}
\author{Martial Geoffre-Rouland \and Hugo Mougard}
\institute{Journée thématique IA de la Nantes Digital Week}

% Let me use colours by name.
\newcommand\blue[1]{\textcolor{blue}{#1}}
\newcommand\red[1]{\textcolor{red}{#1}}
\newcommand\green[1]{\textcolor{green}{#1}}
\newcommand\gray[1]{\textcolor{gray}{#1}}
\newcommand\purple[1]{\textcolor{purple}{#1}}
\newcommand\smallgray[1]{\textcolor{gray}{\footnotesize\it #1}}
\newcommand\prevwork[1]{\smallgray{#1}}

% Some ways to include images.
% o=only, f=full, h=fit height.
% more g's or h's means smaller still.
\newcommand\cimgo[1]{\vfill\centerline{\includegraphics{#1}}\vfill}
\newcommand\cimgff[1]{\vfill\centerline{\includegraphics[width=1.2\textwidth]{#1}}\vfill}
\newcommand\cimgf[1]{\vfill\centerline{\includegraphics[width=\textwidth]{#1}}\vfill}
\newcommand\cimg[1]{\vfill\centerline{\includegraphics[width=.9\textwidth]{#1}}\vfill}
\newcommand\cimgg[1]{\vfill\centerline{\includegraphics[width=.8\textwidth]{#1}}\vfill}
\newcommand\cimggg[1]{\vfill\centerline{\includegraphics[width=.7\textwidth]{#1}}\vfill}
\newcommand\cimgsm[1]{\vfill\centerline{\includegraphics[width=.4\textwidth]{#1}}\vfill}
\newcommand\cimgt[1]{\centerline{\includegraphics[width=.2\textwidth]{#1}}\vfill}
\newcommand\cimgh[1]{\vfill\centerline{\includegraphics[height=.9\textheight]{#1}}\vfill}
\newcommand\cimghh[1]{\vfill\centerline{\includegraphics[height=.8\textheight]{#1}}\vfill}
\newcommand\cimghhh[1]{\vfill\centerline{\includegraphics[height=.7\textheight]{#1}}\vfill}
\newcommand\cimghhhh[1]{\vfill\centerline{\includegraphics[height=.6\textheight]{#1}}\vfill}

% Just one phrase centered horizontally and vertically.
\newcommand\vphrase[1]{\vfill\centerline{\large\bf\blue{#1}}\vfill}
% Just one phrase, I'll position it.
\newcommand\phrase[1]{\centerline{\huge\bf\blue{#1}}}

\begin{document}
\maketitle

\section{Intro aux réseaux de neurones artificiels}
\label{sec:nn}

\begin{frame}{Réseaux de neurones}
  \textbf{Inspirés} du cerveau :
  \begin{itemize}
  \item 1 neurone = 1 calcul simple
  \item 10M de neurones = 10M de calculs simples
  \item 10M de neurones combinés = 10M de calculs simples combinés = 1
    calcul intéressant
  \end{itemize}
\end{frame}

\begin{frame}{Réseaux de neurones --- Exemple}
  \cimgg{neural-network.png}
\end{frame}

\begin{frame}{Réseaux de neurones --- Neurone biologique}
  \cimgg{neuron.png}
\end{frame}

\begin{frame}{Réseaux de neurones --- Neurone artificiel}
  \cimgg{neuron-model.png}
\end{frame}

\begin{frame}{Réseaux de neurones --- Pourquoi a-t-on besoin
    d'activations ?}
  \cimgg{xor.png}
\end{frame}

\begin{frame}{Réseaux de neurones --- Fonction d'activation: exemple}
  \cimgg{activation.jpeg}
\end{frame}

\begin{frame}{Réseaux de neurones --- C'est quoi déjà un neurone ?}
  \begin{itemize}
  \item   Neuron = Weighted averaged of its input + activation function
  \item   Goal of learning = learn good weights.
  \end{itemize}
\end{frame}

\begin{frame}{Réseaux de neurones --- Exemple (le retour)}
  \cimgg{neural-network.png}
\end{frame}

\begin{frame}{Réseaux de neurones --- Comment savoir si le réseau est
    bon ?}
  \begin{itemize}
  \item Calculer nous-même ce qu'essaye de calculer le réseau
  \item Comparer avec ce qu'il sort (fonction de perte).
  \end{itemize}
\end{frame}

\begin{frame}{Réseaux de neurones --- Comment apprendre les poids ?}
  \begin{enumerate}
  \item Une fonction de perte est une \textbf{fonction}
  \item On voit au lycée comment minimiser des fonctions (dérivées)
  \item …
  \item Profit
  \end{enumerate}
\end{frame}

\section{Convolutionnal Neural Networks}
\label{sec:convnets}
\begin{frame}{CNNs --- Les réseaux normaux ne suffisent pas ?}
  \vphrase{Pas pour les images. Chaque neurone reçoit chaque pixel.}
  \vphrase{→ Trop d'informations !}
  \vfill
  \cimgg{tmi.jpg}
  \vfill
\end{frame}
\begin{frame}{CNNs --- Trop d'infos 1/3}
  \cimgg{park.jpg}
\end{frame}
\begin{frame}{CNNs --- Trop d'infos 2/3}
  \cimgg{station.jpg}
\end{frame}
\begin{frame}{CNNs --- Trop d'infos 3/3}
  \cimgg{street.jpg}
\end{frame}
\begin{frame}{CNNs --- Solution 1/3}
  \vphrase{Restreindre les champs de réception des neurones}
  \cimggg{kernel.jpg}
\end{frame}
\begin{frame}{CNNs --- Solution 2/3}
  \vphrase{Pooling: propager seulement les infos importantes}
  \cimg{pooling.png}
\end{frame}
\begin{frame}{CNNs --- Solution 3/3}
  \vphrase{Combiner ces mécanismes hiérarchiquement}
  \cimg{cnn.jpg}
\end{frame}
\begin{frame}{CNNs --- Exemples}
  \cimg{hierarchical-features.png}
\end{frame}


\section{Outils}
\label{sec:outils}
\begin{frame}{Outils --- Python}
  \cimgsm{python.png}
  \begin{itemize}
  \item Language de loin le plus utilisé en ML
  \item Facile d'accès
  \item Bonnes perfs car libs écrites en C
  \end{itemize}
\end{frame}
\begin{frame}{Outils --- Jupyter}
  \cimgsm{jupyter.png}
  \begin{itemize}
  \item Outil interactif pour programmer en python
  \item Sympa pour prototyper
  \item Très utilisé pour partager des expés
  \end{itemize}
\end{frame}
\begin{frame}{Outils --- Tensorflow}
  \cimgsm{tensorflow.png}
  \begin{itemize}
  \item Librairie de deep learning de Google
  \item Bon compromis entre recherche et production
  \item Un peu dur d'utilisation
  \end{itemize}
\end{frame}
\begin{frame}{Outils --- PyTorch}
  \cimgsm{pytorch.png}
  \begin{itemize}
  \item TensorFlow version Facebook
  \item Très bien pour la recherche et le prototyping
  \item Plein d'algos dispos sur github
  \end{itemize}
\end{frame}
\begin{frame}{Outils --- Keras}
  \cimgt{keras.jpg}
  \begin{itemize}
  \item Deep learning
  \item Intégré à TensorFlow
  \item Python
  \item Rend TF plus facile
  \end{itemize}
\end{frame}
\begin{frame}{Outils --- Scikit Learn}
  \cimgsm{scikit.png}
  \begin{itemize}
  \item Machine learning \og traditionnel\fg{}
  \item Énormément d'algos dispos
  \item Python
  \end{itemize}
\end{frame}
\begin{frame}{Outils --- Google Cloud}
  \cimgsm{google-cloud.png}
  \begin{itemize}
  \item GPUs peu chers et sur mesure
  \item 300€ de crédit offert
  \end{itemize}
\end{frame}

\end{document}
%%% Local Variables:
%%% mode: latex
%%% TeX-master: t
%%% End:
